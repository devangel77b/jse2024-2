\documentclass[letterpaper,11pt,twoside,electronic,headers=exceptpdf,papers=countpages,linkcolor=blue,urlcolor=blue]{confproc}

\renewcommand{\procpdfauthor}{Dennis Evangelista, Manalapan High School}
\renewcommand{\procpdftitle}{Journal of Science \& Engineering}
\renewcommand{\procchead}{}
\renewcommand{\proclhead}{}
\renewcommand{\proccfoot}{\thepage}

\author{\procpdfauthor}
\title{\procpdftitle}
\date{\today}
\renewcommand{\PAPERPATH}{.}

\makeindex

\usepackage{pdfpages}
\usepackage{helvet,mathptmx}
\renewcommand{\contentsname}{{\normalfont\sffamily Journal of\medskip}\\{\normalfont\Huge Science \& Engineering}}
\usepackage{verse}

\begin{document}
\frontmatter
\setcounter{page}{1}
\pdfbookmark[0]{Preamble}{preamble}
\pdfbookmark[1]{Cover}{cover}
%\maketitle
\index{Thaker, Pooja}\includepdf[pages=1]{cover}
\newpage
\noindent Journal of Science Engineering (J S\&E) publishes high quality original research, short articles, articles, letters, and reports by promising students of science and engineering.  Our impact factor is 20 and our $h$-index is 200.  J S\&E is a publication of the Freehold Regional High School District Press. 
\vspace{2ex}

%\masthead here
\noindent Our editorial office is located at:\\

\noindent Journal of Science \& Engineering\\20 Church Lane Room G201\\Englishtown, NJ 07726\\United States
\vspace{2ex}

\noindent\small Editorial board
\begin{itemize}
\itemsep-0.5em
\item[] E Carteya
\item[] D Evangelista
\item[] E Levin
\item[] V Oleksy
\item[] S Pepper
\item[] J Philhower
\item[] J Powers
\item[] K Suchodolski
\item[] H Vyas
\end{itemize}

\noindent\small Advisory board
\begin{itemize}
\itemsep-0.5em
\item[] S Currie
\item[] A Brusotti
\item[] J Hein
\item[] M Venuto
\item[] B English
\item[] J Komitas
\end{itemize}

\vfill

\noindent{Copyright \copyright\ 2024 The Journal of Science and Engineering. This work is licensed under CC BY-NC 4.0. To view a copy of this license, visit \url{https://creativecommons.org/licenses/by-nc/4.0/}}
\vspace{2ex}

\noindent{\url{https://manalapan.frhsd.com/}}
\vspace{2ex}

\noindent{\url{https://sites.google.com/frhsd.com/seopenhouse/home}}
\vspace{2ex}

\otherpagestyle
\addtocontents{toc}{Volume 1, Number 2, December 11, 2024\bigskip\\}
\tableofcontents
\addtocontents{toc}{\emph{From the cover:} The electric potential, a scalar quantity, can be visualized by drawing equipotentials; isolines or level curves in which potential is a constant. The electric field, a vector field, points in a direction perpendicular to the equipotentials, down the gradient. These physics concepts might be important to consider when designing equipment that makes use of electric fields, such as in a low-cost gel electrophoresis rig. Here, researchers consider this problem. Electrophoresis is a technique where molecules can be sorted according to their charge and size. By characterizing the field and equipotentials within a prototype device, one could eventually predict the movement of various biomolecules of interest, such as fragments of DNA cut into parts at specific nucleotide sequences by restriction enzymes. \emph{Cover image: Pooja Thaker.}\bigskip\\}




% Haiku here
\vspace*{\fill}
\addcontentsline{toc}{section}{Inside J S\&E\\\emph{Vikram Choudhury}}
\begin{verse}
Equipotential\\
Square to the electric field\\
Charges move without work
\end{verse}
\vspace{2ex}{\raggedleft{Vikram Choudhury}\par}
\index{Choudhury, Vikram}
\vspace*{\fill}




\mainmatter
\procpaper[%
title={Computational mapping analysis of equipotential and electric field lines in gel electrophoresis rig},
author={Saketh Ayyagari, Victoria Collemi, Krish Shah, and Kevin Tomazic},
index={\index{Ayyagari, Saketh}\index{Collemi, Victoria}\index{Shah, Krish}\index{Tomazic, Kevin}},
]{/ayyagari/ayyagari}

\procpaper[%
title={Demonstrating a method to create a low-cost electrophoresis rig solution},
author={Srikar Baru, Vikram Choudhury, Pooja Thaker, Nathan Martin, and Danyal Ahmad},
index={\index{Baru, Srikar}\index{Choudhury, Vikram}\index{Thaker, Pooja}\index{Martin, Nathan}\index{Ahmad, Danyal}},
]{/baru/baru}

\procpaper[%
title={An experimental study on electric potential and equipotential regions},
author={Ryan Cohen, Shreyas Musuku, Justin Hammer, Eshan Handique, Nirvik Patel, Dilan Gandhi, and Nathan Gershteyn},
index={\index{Cohen, Ryan}\index{Musuku, Shreyas}\index{Hammer, Justin}\index{Handique, Eshan}\index{Patel, Nirvik}\index{Gandhi, Dilan}\index{Gershteyn, Nathan}},
]{/cohen/cohen}

\procpaper[%
title={Electric field mapping for cost-effective gel electrophoresis applications},
author={Ryan Edwards, Cameron Karabin, Chloe Li, Krish Patel, and Jake Schatz},
index={\index{Edwards, Ryan}\index{Karabin, Cameron}\index{Li, Chloe}\index{Patel, Krish}\index{Schatz, Jake}},
]{/edwards/edwards}

\procpaper[%
title={Mapping electric potential and electric field distribution in saltwater and investigating the effect of distance from source},
author={Aadarsh Kumar, Steven Perkins, Nareshanjay Muthukumar, Henry Villase\~{n}or, and Anirudh Khanna},
index={\index{Kumar, Aadarsh}\index{Perkins, Steven}\index{Muthukumar, Nareshanjay}\index{Villasenor, Henry}\index{Khanna, Anirudh}},
]{/kumar/kumar}

\backmatter
%\bibliographystyle{plain}
%\bibliography{\procbibfile}

\insertindex

% get on even page?!
%\includepdf[pages=1]{main-confproc-old.pdf}
\end{document}